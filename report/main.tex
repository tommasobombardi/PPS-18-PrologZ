%%%%%%%%%%%%%%%%%%%%%%%%%%%%%%%%%%%%%%%%%
% Masters/Doctoral Thesis 
% LaTeX Template
% Version 2.5 (27/8/17)
%
% This template was downloaded from:
% http://www.LaTeXTemplates.com
%
% Version 2.x major modifications by:
% Vel (vel@latextemplates.com)
%
% This template is based on a template by:
% Steve Gunn (http://users.ecs.soton.ac.uk/srg/softwaretools/document/templates/)
% Sunil Patel (http://www.sunilpatel.co.uk/thesis-template/)
%
% Template license:
% CC BY-NC-SA 3.0 (http://creativecommons.org/licenses/by-nc-sa/3.0/)
%
%%%%%%%%%%%%%%%%%%%%%%%%%%%%%%%%%%%%%%%%%

%----------------------------------------------------------------------------------------
%	PACKAGES AND OTHER DOCUMENT CONFIGURATIONS
%----------------------------------------------------------------------------------------

\documentclass[
12pt, % The default document font size, options: 10pt, 11pt, 12pt
oneside, % Two side (alternating margins) for binding by default, uncomment to switch to one side
italian, % ngerman for German
onehalfspacing, % Single line spacing, alternatives: onehalfspacing or doublespacing
%draft, % Uncomment to enable draft mode (no pictures, no links, overfull hboxes indicated)
%nolistspacing, % If the document is onehalfspacing or doublespacing, uncomment this to set spacing in lists to single
%liststotoc, % Uncomment to add the list of figures/tables/etc to the table of contents
%toctotoc, % Uncomment to add the main table of contents to the table of contents
%parskip, % Uncomment to add space between paragraphs
%nohyperref, % Uncomment to not load the hyperref package
headsepline, % Uncomment to get a line under the header
%chapterinoneline, % Uncomment to place the chapter title next to the number on one line
%consistentlayout, % Uncomment to change the layout of the declaration, abstract and acknowledgements pages to match the default layout
]{structure} % The class file specifying the document structure

\usepackage[utf8]{inputenc} % Required for inputting international characters
\usepackage[T1]{fontenc} % Output font encoding for international characters
\usepackage{listings}
\linespread{1.25}

\usepackage{mathpazo} % Use the Palatino font by default
\usepackage{amsmath}
\usepackage{amssymb}

\usepackage[backend=bibtex,style=numeric,natbib=true,sorting=none]{biblatex} % Use the bibtex backend with the authoryear citation style (which resembles APA)

\addbibresource{biblio.bib} % The filename of the bibliography

\usepackage[autostyle=true]{csquotes} % Required to generate language-dependent quotes in the bibliography

%----------------------------------------------------------------------------------------
%	MARGIN SETTINGS
%----------------------------------------------------------------------------------------

\geometry{
	paper=a4paper, % Change to letterpaper for US letter
	inner=2.5cm, % Inner margin
	outer=3.8cm, % Outer margin
	bindingoffset=.5cm, % Binding offset
	top=1.5cm, % Top margin
	bottom=1.5cm, % Bottom margin
	%showframe, % Uncomment to show how the type block is set on the page
}

%----------------------------------------------------------------------------------------
%	THESIS INFORMATION
%----------------------------------------------------------------------------------------

\thesistitle{PPS-18-PrologZ} % Your thesis title, this is used in the title and abstract, print it elsewhere with \ttitle
\examiner{} % Your examiner's name, this is not currently used anywhere in the template, print it elsewhere with \examname
\degree{Doctor of Philosophy} % Your degree name, this is used in the title page and abstract, print it elsewhere with \degreename
\author{Bombardi Tommaso} % Your name, this is used in the title page and abstract, print it elsewhere with \authorname
\addresses{} % Your address, this is not currently used anywhere in the template, print it elsewhere with \addressname

\subject{Biological Sciences} % Your subject area, this is not currently used anywhere in the template, print it elsewhere with \subjectname
\keywords{} % Keywords for your thesis, this is not currently used anywhere in the template, print it elsewhere with \keywordnames
\university{{Università di Bologna}} % Your university's name and URL, this is used in the title page and abstract, print it elsewhere with \univname
\department{{Dipartimento}} % Your department's name and URL, this is used in the title page and abstract, print it elsewhere with \deptname
\group{\href{http://researchgroup.university.com}{Research Group Name}} % Your research group's name and URL, this is used in the title page, print it elsewhere with \groupname
\faculty{Corso di Laurea Magistrale in Ingegneria e Scienze Informatiche} % Your faculty's name and URL, this is used in the title page and abstract, print it elsewhere with \facname

\AtBeginDocument{
\hypersetup{pdftitle=\ttitle} % Set the PDF's title to your title
\hypersetup{pdfauthor=\authorname} % Set the PDF's author to your name
\hypersetup{pdfkeywords=\keywordnames} % Set the PDF's keywords to your keywords
}

\begin{document}

\frontmatter % Use roman page numbering style (i, ii, iii, iv...) for the pre-content pages

\pagestyle{plain} % Default to the plain heading style until the thesis style is called for the body content

\lstset{ 
    language=Scala, % choose the language of the code
    basicstyle=\fontfamily{pcr}\selectfont\footnotesize\color{black},
    keywordstyle=\color{blue}\bfseries, % style for keywords
    numbers=none, % where to put the line-numbers
    numberstyle=\tiny, % the size of the fonts that are used for the line-numbers     
    backgroundcolor=\color{white},
    showspaces=false, % show spaces adding particular underscores
    showstringspaces=false, % underline spaces within strings
    showtabs=false, % show tabs within strings adding particular underscores
    frame=single, % adds a frame around the code
    tabsize=2, % sets default tabsize to 2 spaces
    rulesepcolor=\color{gray},
    rulecolor=\color{black},
    captionpos=b, % sets the caption-position to bottom
    breaklines=true, % sets automatic line breaking
    breakatwhitespace=false
}

\renewcommand{\lstlistingname}{Codice}% Listing -> Codice

%----------------------------------------------------------------------------------------
%	TITLE PAGE
%----------------------------------------------------------------------------------------

\begin{titlepage}
\begin{center}

{\scshape\LARGE \univname\par}\vspace{0.8cm} % University name
{\scshape\LARGE \facname\par}\vspace{1.8cm} % University name
\textsc{\Large Paradigmi di Programmazione e Sviluppo}\\[1.2cm] % Thesis type

\HRule \\[0.4cm] % Horizontal line
{\huge \bfseries \ttitle\par}\vspace{0.4cm} % Thesis title
\HRule \\[1.5cm] % Horizontal line
 
\begin{minipage}[t]{0.4\textwidth}
\begin{flushleft} \large
\end{flushleft}
\end{minipage}
\begin{minipage}[t]{0.4\textwidth}
\begin{flushright} \large
\end{flushright}
\end{minipage}\\[3cm]
 
\vfill

\LARGE\authorname \\[0.8cm] % University requirement text

\vfill

{\large \today}\\[4cm] % Date
%\includegraphics{Logo} % University/department logo - uncomment to place it
 
\vfill
\end{center}
\end{titlepage}

%----------------------------------------------------------------------------------------
%	LIST OF CONTENTS/FIGURES/TABLES PAGES
%----------------------------------------------------------------------------------------

\tableofcontents % Prints the main table of contents

%----------------------------------------------------------------------------------------
%	THESIS CONTENT - CHAPTERS
%----------------------------------------------------------------------------------------

\mainmatter % Begin numeric (1,2,3...) page numbering

\pagestyle{thesis} % Return the page headers back to the "thesis" style

% Include the chapters of the thesis as separate files from the Chapters folder
% Uncomment the lines as you write the chapters

\chapter{Scalaz} % Main chapter title

\label{Chapter1} % Change X to a consecutive number; for referencing this chapter elsewhere, use \ref{ChapterX}

Scalaz\footnote{\url{https://github.com/scalaz/scalaz}} è una libreria per la programmazione funzionale, che mette a disposizione un grande numero di strutture dati puramente funzionali create con lo scopo di integrare quelle presenti nella Scala Standard Library. In Scalaz sono infatti presenti diversi nuovi elementi, definiti in genere sotto forma di typeclass, ma anche delle estensioni per le strutture dati standard.


%----------------------------------------------------------------------------------------
%	SECTION 1
%----------------------------------------------------------------------------------------

\section{Sezione 1}

%----------------------------------------------------------------------------------------
%	SECTION 2
%----------------------------------------------------------------------------------------

\section{Sezione 2}

%----------------------------------------------------------------------------------------
%	SECTION 3
%----------------------------------------------------------------------------------------

\section{Sezione 3}

\chapter{DSL for Logic Programming} % Main chapter title

\label{Chapter2} % Change X to a consecutive number; for referencing this chapter elsewhere, use \ref{ChapterX}

Questo secondo capitolo sarà dedicato alla presentazione del caso d'uso della libreria Scalaz realizzato in questo elaborato. Si è scelto di sfruttare alcune delle astrazioni offerte da Scalaz per implementare un DSL che imiti Prolog, in modo da fornire un supporto diretto alla programmazione logica.

Il DSL ha l'obiettivo di offrire un'interfaccia quanto più possibile semplice e simile a Prolog, ma che consenta di scrivere programmi direttamente in Scala. L'esecuzione di questi ultimi sarà basata sulla tecnica di risoluzione comunemente usata dai motori logici, ossia quella che prevede la costruzione di un SLD tree, e anch'essa sarà implementata con l'ausilio di Scalaz.

%----------------------------------------------------------------------------------------
%	SECTION 1
%----------------------------------------------------------------------------------------

\section{Available features}

Il DSL realizzato supporta una parte di quanto offerto dal linguaggio Prolog e, di seguito, sono riportate le funzionalità che mette a disposizione:

\begin{itemize}
\item Numeri di tipo \textit{Int} e \textit{Double} possono essere utilizzati per rappresentare rispettivamente i numeri interi e decimali presenti in Prolog.
\item Stringhe che iniziano con una lettera minuscola corrispondono a dei termini costanti, mentre le stringhe il cui primo carattere è costituito da una lettera maiuscola equivalgono a delle variabili.
\item Mentre in fase di compilazione tutte le stringhe vengono accettate in quanto termini, a run-time non sono considerate valide le stringhe vuote e quelle che non contengono solo lettere. Quest'ultimo vincolo è stato aggiunto per colmare il gap tra un linguaggio che è tipato staticamente (Scala) e uno che non lo è (Prolog). Sono infatti evitati i casi in cui si confronta un numero (ad esempio 1) e la stringa corrispondente (``1''), che in Prolog non possono verificarsi. Inoltre, escludere i caratteri speciali consente alle stringhe di non contenere delimitatori come ``,'' e ``)''.
\item È possibile creare un funtore usando il metodo \textit{apply} dell'oggetto \textit{Struct} e specificando il nome del funtore, che deve essere una stringa non vuota, contenere solo lettere e iniziare con una lettera minuscola.
\item A sua volta, un funtore offre un metodo \textit{apply} che accetta un numero variabile di termini e restituisce un termine composto. È previsto un errore se non viene passato nessun termine quando si usa un funtore. Se si divide la definizione del funtore dalla creazione del termine composto, questa operazione è molto simile a ciò che si farebbe in Prolog.

\lstinputlisting{code/PrologCompound.scala}
\item In modo speculare a quanto visto per funtori e termini composti, il DSL offre l'oggetto \textit{Predicate} con cui è possibile inizializzare dei predicati che a loro volta sono utilizzabili per la creazione di fatti.
\item Sono supportate le regole, che possono essere definite a partire da un fatto chiamando il metodo :- e passando ad esso un numero variabile di fatti (previsto un errore in assenza di argomenti). Il fatto su cui viene chiamato il metodo costituisce la testa della regola, mentre gli argomenti corrispondono al suo corpo. La notazione infissa disponibile in Scala rende naturale la definizione di regole, ma è importante ricordare che si tratta di una chiamata di metodo e quindi, se il corpo della regola contiene più fatti, è necessario racchiuderli tra parentesi.

\lstinputlisting{code/PrologRule.scala}
\item Gli elementi descritti finora consentono di definire un programma Prolog, che è costituito da una serie di clausole corrispondenti alla teoria (che possono essere fatti o regole) e da un insieme di fatti che rappresentano i goal. Per la loro risoluzione il DSL fornisce un oggetto \textit{Engine}, che costituisce il motore logico condiviso nell'applicazione.
\item \textit{Engine} consente di aggiungere e resettare la teoria del programma, tramite i metodi \textit{addTheory} e \textit{resetTheory}, e fornisce due modalità di risoluzione alternative (le stesse offerte da tuProlog), grazie ai metodi \textit{solve} e \textit{solveAll}, con cui è possibile risolvere uno o più goal. Nella prima modalità la computazione procede finché non viene trovata una soluzione e, in quel momento, l'utente può scegliere se interrompere l'esecuzione o proseguire fino alla soluzione successiva. Nella seconda modalità, invece, sono esplorate tutte le possibili alternative. È inoltre possibile abilitare la stampa dell'albero di esecuzione del programma.
\item Come si può facilmente intuire, ogni chiamata di \textit{solve} o \textit{solveAll} corrisponde all'esecuzione di un programma Prolog. Per ognuna di esse \textit{Engine} porta a termine un procedimento di validazione dell'input, che prevede la mancata esecuzione del programma in caso di errori nei suoi elementi (variabili, termini costanti, termini composti, fatti o regole) e la restituzione di un report contenente tutti gli errori trovati.
\end{itemize}

Per utilizzare il DSL è sufficiente specificare gli statement \textit{import prologz.dsl.\_}, con cui si ottengono tutti i costrutti visti in precedenza, \textit{import prologz.dsl.ClauseImplicits.\_} e \textit{import prologz.dsl.TermImplicits.\_}, che mettono a disposizione le conversioni implicite necessarie per la definizione di termini, fatti e regole.

%----------------------------------------------------------------------------------------
%	SECTION 2
%----------------------------------------------------------------------------------------

\section{Implementation details}

Il codice realizzato è suddiviso in due package: \textit{prologz.dsl}, che contiene la definizione di tutti gli elementi di Prolog supportati dal DSL e gli strumenti messi a disposizione degli utilizzatori del DSL, e \textit{prologz.resolution}, destinato a ciò che è necessario per la risoluzione dei programmi Prolog.

\subsection{Program validation}

Per quanto riguarda i costrutti di Prolog supportati dal DSL, si è scelto di modellare i termini come case class che definiscono dei sottotipi del trait \textit{Term}: \textit{Atom[A]} per i termini costanti e i numeri (interi o decimali), \textit{Variable} per le variabili e \textit{Struct} per i termini composti. Allo stesso modo, le case class \textit{Fact} e \textit{Rule} corrispondono ai fatti e alle regole ed estendono il trait \textit{Clause}. Funtori e predicati sono invece rappresentati tramite stringhe taggate, rispettivamente \textit{String @@ Functor} e \textit{String @@ Predicate}, poiché si tratta semplicemente di elementi che devono essere in grado di incapsulare una stringa.

Le modalità standard per la costruzione di valori di questi tipi sono però nascoste agli utilizzatori del DSL, che hanno a disposizione solo quanto visto nella sezione precedente per definire gli elementi del programma Prolog. Il motivo di questa scelta è evidente se si considerano le definizioni dei metodi \textit{addTheory}, \textit{solve} e \textit{solveAll} presenti in \textit{Engine}: essi accettano rispettivamente \textit{PzValidation[Clause]*} e \textit{PzValidation[Fact]*} (invece di \textit{Clause*} e \textit{Fact*}).

\textit{PzValidation[A]} è un semplice type alias che corrisponde a \textit{ValidationNel[String @@ InputError, A]}, dove String @@ InputError è un tipo usato per rappresentare gli errori nella definizione del programma Prolog. Tutti gli elementi presenti al suo interno, a partire dai più semplici come termini costanti e variabili, sono creati come istanze di \textit{PzValidation[A]} (A è il tipo che rappresenta quell'elemento) e possono quindi incapsulare errori nella loro definizione.

In questo modo, sfruttando la possibilità di accumulare gli errori offerta dal costrutto \textit{Validation} e la sintassi proposta dall'\textit{Applicative Builder} per gestire i casi di successo, è stato realizzato il meccanismo di validazione dell'input descritto in precedenza. In caso di errori viene semplicemente restituito un report, evitando di lanciare eccezioni e di interrompere l'esecuzione.

Un altro vantaggio si ha nel caso di più programmi Prolog da eseguire: se sono presenti errori in alcuni di essi, gli altri vengono risolti normalmente.

\lstinputlisting[caption=Metodi per la creazione di \textit{PzValidation[Term]} e per la concatenazione di \textit{PzValidation[Clause]} e di \textit{PzValidation[Fact]}]{code/PrologValidation.scala}

\subsection{Prolog execution}

L'esecuzione dei programmi Prolog è basata sulla costruzione dell'SLD tree, un albero in cui ogni nodo corrisponde ad una coppia contenente sia i goal che devono ancora essere risolti sia una sostituzione che rappresenta una soluzione valida per i goal che sono stati risolti fino a quel momento.

Nella programmazione logica, una sostituzione è definita come una mappa finita da variabili a termini e, quindi, si è scelto di rappresentarla tramite l'alias \textit{type Substitution = Map[Variable, Term]}. Il meccanismo di risoluzione di Prolog prevede che più sostituzioni debbano poter essere unite in un'unica sostituzione avente come effetto l'esecuzione sequenziale delle varie sostituzioni considerate: dati due goal R1 e R2 risolti rispettivamente dalle sostituzioni A e B, la loro soluzione sarà rappresentata da una sostituzione AB con effetto equivalente all'applicazione di A seguita dall'applicazione di B.

La politica di combinazione delle mappe offerta da Scalaz con l'implementazione di \textit{Monoid[Map]} non è adatta alle sostituzioni, perché le unisce semplicemente e sovrascrive i valori che compaiono più volte. Ad esempio, date due sostituzioni A=\{X/2,Y/X\} e B=\{X/3\}, si ottiene come risultato AB=\{Y/X,X/3\} mentre la sostituzione equivalente è AB=\{X/2,Y/3\}.

Perciò è stata definita una nuova implementazione di \textit{Monoid[Substitution]}, in cui aggiungere una sostituzione ad un'altra equivale ad applicarla a tutti i termini nelle relazioni variabile/termine della prima sostituzione. In questo modo è sufficiente usare l'operatore |+| per unire le sostituzioni ma, affinché il procedimento funzioni, è necessario che la sostituzione inziale sia un'identità in cui ogni variabile presente nei goal è associata a sé stessa.

\lstinputlisting[caption={Definizione di \textit{SubstitutionMonoid}, il metodo \textit{substitute} consente (utilizzando un implicito) di applicare una sostituzione ad una lista di termini}]{code/PrologSubstitution.scala}

Dopo aver stabilito come rappresentare e unire le sostituzioni, è stato possibile implementare il meccanismo di risoluzione. Esso prevede che tutti i programmi Prolog siano convertiti in alberi SLD, realizzati e gestiti sfruttando le strutture dati \textit{Tree} e \textit{TreeLoc} offerte da Scalaz. Per ogni nodo dell'albero, oltre ai goal da risolvere e alla sostituzione eseguita, è memorizzata una lista di clausole che rappresentano la parte di teoria non ancora considerata.

Inizialmente, l'albero sarà costituito da un solo nodo contenente la teoria del programma, i goal e la sostituzione identità.

\lstinputlisting[caption=a]{code/PrologTree.scala}

\subsection{Unification}


%----------------------------------------------------------------------------------------
%	BIBLIOGRAPHY
%----------------------------------------------------------------------------------------

\printbibliography[heading=bibintoc]

%----------------------------------------------------------------------------------------

\end{document}  
