\chapter{Scalaz} % Main chapter title

\label{Chapter1} % Change X to a consecutive number; for referencing this chapter elsewhere, use \ref{ChapterX}

Scalaz\footnote{\url{https://github.com/scalaz/scalaz}} è una libreria per la programmazione funzionale, che mette a disposizione un grande numero di strutture dati puramente funzionali create con lo scopo di integrare quelle presenti nella Scala Standard Library. In Scalaz sono infatti presenti diversi nuovi elementi, definiti in genere sotto forma di typeclass, ma anche delle estensioni per le strutture dati standard.


%----------------------------------------------------------------------------------------
%	SECTION 1
%----------------------------------------------------------------------------------------

\section{Sezione 1}

%----------------------------------------------------------------------------------------
%	SECTION 2
%----------------------------------------------------------------------------------------

\section{Sezione 2}

%----------------------------------------------------------------------------------------
%	SECTION 3
%----------------------------------------------------------------------------------------

\section{Sezione 3}
